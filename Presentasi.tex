\documentclass{beamer}
\usetheme{Madrid}
\usecolortheme{default}

\usepackage{graphicx}
\usepackage{booktabs}
%\usepackage[indonesian]{babel} % Disabled to prevent PGF conflict

\title[Analisis GWR COVID-19 Inggris]{ANALISIS GEOGRAPHICALLY WEIGHTED REGRESSION (GWR) PADA KASUS COVID-19 DI INGGRIS}
\subtitle{MAKALAH}
\author[Ammar, Norman, Kirono, Devana]{
    Ammar Hanafi (2206051582) \texorpdfstring{\\}{, }
    Norman Mowlana Aziz (2206025470) \texorpdfstring{\\}{, }
    Kirono Dwi Saputro (2106656365) \texorpdfstring{\\}{, }
    Devana Solea (2306262402)
}
\institute[Univ. Indonesia]{
    Program Studi Sarjana Statistika \\
    Fakultas Matematika dan Ilmu Pengetahuan Alam \\
    Universitas Indonesia
}
\date{Juni 2026}

\logo{\includegraphics[height=1cm]{figs/makara.png}}

\begin{document}

\frame{\titlepage}

\begin{frame}{Daftar Isi}
    \tableofcontents
\end{frame}

\section{Pendahuluan}
\begin{frame}{Latar Belakang dan Tujuan}
    \textbf{Latar Belakang}
    \begin{itemize}
        \item COVID-19 memiliki dampak spasial yang tidak seragam.
        \item Faktor risiko (etnis, kesehatan) mungkin memiliki pengaruh yang berbeda di setiap wilayah (Heterogenitas Spasial).
        \item Regresi Global (OLS) mengasumsikan parameter konstan, yang mungkin bias.
    \end{itemize}
    \vspace{0.5cm}
    \textbf{Tujuan}
    \begin{itemize}
        \item Memodelkan tingkat kasus COVID-19 di Inggris.
        \item Menganalisis pengaruh Deprivasi (IMD), Penyakit, dan Kepadatan Hunian.
        \item Membandingkan model OLS dan GWR.
        \item Memetakan variasi koefisien lokal.
    \end{itemize}
\end{frame}

\section{Metodologi}
\begin{frame}{Data dan Variabel}
    \textbf{Unit Analisis}: Upper Tier Local Authorities (n=149) di Inggris.
    \vspace{0.5cm}
    \begin{table}
        \centering
        \begin{tabular}{ll}
            \toprule
            \textbf{Variabel} & \textbf{Deskripsi} \\
            \midrule
            $Y$ (Dependen) & Rate COVID-19 per 100k penduduk \\
            $X_1$ (Independen) & Index of Multiple Deprivation (IMD) \\
            $X_2$ (Independen) & Proporsi Penyakit Jangka Panjang \\
            $X_3$ (Independen) & Proporsi Rumah Tangga Padat (Crowded) \\
            \bottomrule
        \end{tabular}
    \end{table}
    \vspace{0.2cm}
    \small \textit{*Seluruh variabel dilakukan standardisasi (Z-score) sebelum pemodelan agar koefisien dapat diperbandingkan.}
\end{frame}

\begin{frame}{Metode Analisis}
    \begin{enumerate}
        \item \textbf{Analisis Deskriptif}: Eksplorasi data dan peta sebaran.
        \item \textbf{Global OLS}: $Y = \beta_0 + \beta_1 \text{IMD} + \beta_2 \text{Illness} + \beta_3 \text{Crowded} + \epsilon$
        \item \textbf{Uji Asumsi}: Breusch-Pagan Test untuk heteroskedastisitas.
        \item \textbf{GWR (Geographically Weighted Regression)}:
        \[ y_i = \beta_0(u_i,v_i) + \sum \beta_k(u_i,v_i) x_{ik} + \epsilon_i \]
        Menggunakan kernel \textit{Adaptive Bisquare}.
        \item \textbf{Diagnostik}: Local Condition Number (Multikolinearitas).
    \end{enumerate}
\end{frame}

\begin{frame}{Wilayah Studi dan Data Referensi}
    \begin{columns}
        \column{0.5\textwidth}
        \centering
        \includegraphics[width=1.0\textwidth]{figs/map_boundaries.png} \\
        \tiny Peta Batas Wilayah \& Koordinat
        \column{0.5\textwidth}
        \tiny
        \textbf{Sampel Data (Lokasi \& Variabel)}:
        \begin{tabular}{llrrrrrr}
\toprule
 & Region Name & Longitude & Latitude & Covid Rate & IMD & Illness & Crowded \\
\midrule
0 & Hartlepool & -1.2702 & 54.6761 & 8.6930 & 35.0370 & 0.2316 & 0.0094 \\
1 & Middlesbrough & -1.2110 & 54.5447 & 14.4496 & 40.4600 & 0.2086 & 0.0172 \\
2 & Redcar and Cleveland & -1.0061 & 54.5675 & 3.6989 & 29.7920 & 0.2272 & 0.0078 \\
3 & Stockton-on-Tees & -1.3066 & 54.5569 & 7.8284 & 25.7900 & 0.1900 & 0.0093 \\
4 & Darlington & -1.5683 & 54.5353 & 6.6310 & 25.6570 & 0.1963 & 0.0085 \\
5 & Halton & -2.6885 & 53.3342 & 7.9525 & 32.3250 & 0.2136 & 0.0098 \\
6 & Warrington & -2.5617 & 53.3916 & 14.3402 & 18.9420 & 0.1731 & 0.0085 \\
7 & Blackburn with Darwen & -2.4636 & 53.7008 & 10.8483 & 36.0130 & 0.2023 & 0.0262 \\
8 & Blackpool & -3.0220 & 53.8216 & 5.6312 & 45.0390 & 0.2557 & 0.0094 \\
9 & Kingston upon Hull, City of & -0.3038 & 53.7692 & 10.9202 & 40.5640 & 0.1965 & 0.0163 \\
\bottomrule
\end{tabular}

    \end{columns}
\end{frame}

\section{Hasil dan Pembahasan}
\begin{frame}{Eksplorasi Data}
    \begin{columns}
        \column{0.5\textwidth}
        \centering
        \includegraphics[width=0.9\textwidth]{figs/map_covid19_r.png} \\
        \tiny Sebaran Kasus COVID-19
        \column{0.5\textwidth}
        \centering
        \includegraphics[width=0.9\textwidth]{figs/map_imd.png} \\
        \tiny Sebaran Skor IMD (Deprivasi)
    \end{columns}
    \vspace{0.3cm}
    \centering
    \tiny Pola spasial terlihat mengelompok (cluster), mengindikasikan autokorelasi spasial.
\end{frame}

\begin{frame}{Korelasi Antar Variabel}
    \begin{columns}
        \column{0.5\textwidth}
        \centering
        \includegraphics[width=0.9\textwidth]{figs/pairplot.png} \\
        \tiny Pairplot Distribusi Variabel
        \column{0.5\textwidth}
        \centering
        \includegraphics[width=0.9\textwidth]{figs/heatmap.png} \\
        \tiny Matriks Korelasi
    \end{columns}
\end{frame}

\begin{frame}{Hasil Model Global (OLS)}
    \begin{table}
        \centering
        \tiny
        
\begin{tabular}{lclc}
\toprule
\textbf{Dep. Variable:}    &    covid19\_r    & \textbf{  R-squared:         } &     0.019   \\
\textbf{Model:}            &       OLS        & \textbf{  Adj. R-squared:    } &     0.006   \\
\textbf{Method:}           &  Least Squares   & \textbf{  F-statistic:       } &     1.432   \\
\textbf{Date:}             & Sun, 08 Feb 2026 & \textbf{  Prob (F-statistic):} &    0.242    \\
\textbf{Time:}             &     11:41:04     & \textbf{  Log-Likelihood:    } &   -404.24   \\
\textbf{No. Observations:} &         149      & \textbf{  AIC:               } &     814.5   \\
\textbf{Df Residuals:}     &         146      & \textbf{  BIC:               } &     823.5   \\
\textbf{Df Model:}         &           2      & \textbf{                     } &             \\
\textbf{Covariance Type:}  &    nonrobust     & \textbf{                     } &             \\
\bottomrule
\end{tabular}
\begin{tabular}{lcccccc}
                     & \textbf{coef} & \textbf{std err} & \textbf{t} & \textbf{P$> |$t$|$} & \textbf{[0.025} & \textbf{0.975]}  \\
\midrule
\textbf{const}       &       4.3646  &        2.209     &     1.975  &         0.050        &       -0.002    &        8.731     \\
\textbf{ethnic}      &       3.7597  &        2.222     &     1.692  &         0.093        &       -0.631    &        8.150     \\
\textbf{lt\_illness} &      10.6082  &       11.128     &     0.953  &         0.342        &      -11.385    &       32.602     \\
\bottomrule
\end{tabular}
\begin{tabular}{lclc}
\textbf{Omnibus:}       & 39.290 & \textbf{  Durbin-Watson:     } &    1.687  \\
\textbf{Prob(Omnibus):} &  0.000 & \textbf{  Jarque-Bera (JB):  } &   76.114  \\
\textbf{Skew:}          &  1.192 & \textbf{  Prob(JB):          } & 2.97e-17  \\
\textbf{Kurtosis:}      &  5.565 & \textbf{  Cond. No.          } &     38.9  \\
\bottomrule
\end{tabular}
%\caption{OLS Regression Results}


Notes: Standard Errors assume that the covariance matrix of the errors is correctly specified.
    \end{table}
    \begin{itemize}
        \item $R^2$ rendah (0.038).
        \item Hanya variabel IMD yang signifikan pada taraf nyata 5\%.
        \item Model global kurang mampu menjelaskan variasi data.
    \end{itemize}
\end{frame}

\begin{frame}{Bukti Heterogenitas Spasial}
    \begin{columns}
        \column{0.5\textwidth}
        \textbf{Uji Breusch-Pagan}
        \begin{itemize}
            \item p-value = 0.8739 ($> 0.05$)
            \item Secara statistik gagal tolak H0.
            \item NAMUN... visualisasi residual berkata lain.
        \end{itemize}
        \column{0.5\textwidth}
        \centering
        \includegraphics[width=0.9\textwidth]{figs/map_ols_residuals.png} \\
        \tiny Peta Residual OLS (Clustered)
    \end{columns}
\end{frame}

\begin{frame}{Perbandingan Model GWR (Bisquare vs Gaussian)}
    \begin{table}
        \centering
        \begin{tabular}{lrrr}
\toprule
 & AICc & R2 & Adj. R2 \\
Model &  &  &  \\
\midrule
OLS & 424.1189 & 0.0377 & 0.0178 \\
GWR Bisquare & 407.4528 & 0.2044 & 0.1579 \\
GWR Gaussian & 408.6839 & 0.1967 & 0.1503 \\
\bottomrule
\end{tabular}

    \end{table}
    \begin{itemize}
        \item \textbf{Adaptive Bisquare} terpilih sebagai model terbaik.
        \item \textbf{AICc Terendah}: 797.0 (vs Gaussian 798.2 vs OLS 813.6).
        \item \textbf{$R^2$ Tertinggi}: 0.204 (vs Gaussian 0.197 vs OLS 0.038).
        \item Model lokal (GWR) jauh lebih unggul dibanding Global.
    \end{itemize}
\end{frame}

\begin{frame}{Perbandingan Residual (Goodness of Fit)}
    \centering
    \includegraphics[width=1.0\textwidth]{figs/residual_comparison.png} \\
    \tiny Perbandingan Residual: OLS (Kiri) vs GWR Bisquare (Tengah) vs GWR Gaussian (Kanan) \\
    \vspace{0.2cm}
    \small
    Residual GWR lebih acak dan variansnya lebih kecil, menandakan model lebih fit.
\end{frame}

\begin{frame}{Peta Koefisien Lokal: IMD}
    \begin{columns}
        \column{0.5\textwidth}
        \centering
        \includegraphics[width=0.9\textwidth]{figs/covid_gwr_imd_coef.png} \\
        \tiny Koefisien
        \column{0.5\textwidth}
        \centering
        \includegraphics[width=0.9\textwidth]{figs/covid_gwr_imd_pval.png} \\
        \tiny P-Value
    \end{columns}
    \vspace{0.2cm}
    \begin{itemize}
        \item P-value rendah (gelap) menunjukkan pengaruh signifikan.
        \item Semakin tinggi deprivasi, semakin tinggi kasus COVID-19 (Koefisien Positif/Merah).
    \end{itemize}
\end{frame}

\begin{frame}{Peta Koefisien Lokal: Penyakit Jangka Panjang}
    \begin{columns}
        \column{0.5\textwidth}
        \centering
        \includegraphics[width=0.9\textwidth]{figs/covid_gwr_illness_coef.png} \\
        \tiny Koefisien
        \column{0.5\textwidth}
        \centering
        \includegraphics[width=0.9\textwidth]{figs/covid_gwr_illness_pval.png} \\
        \tiny P-Value
    \end{columns}
    \vspace{0.2cm}
    \begin{itemize}
        \item Signifikan hanya di sebagian wilayah (cluster tertentu).
        \item Mengungkap hubungan lokal yang tidak terlihat di OLS (Global tidak signifikan).
    \end{itemize}
\end{frame}

\begin{frame}{Peta Koefisien Lokal: Kepadatan Hunian}
    \begin{columns}
        \column{0.5\textwidth}
        \centering
        \includegraphics[width=0.9\textwidth]{figs/covid_gwr_crowded_coef.png} \\
        \tiny Koefisien
        \column{0.5\textwidth}
        \centering
        \includegraphics[width=0.9\textwidth]{figs/covid_gwr_crowded_pval.png} \\
        \tiny P-Value
    \end{columns}
    \vspace{0.2cm}
    \begin{itemize}
        \item Variasi pengaruh kepadatan hunian terlihat jelas.
        \item Signifikansi terbatas pada area spesifik (p-value rendah).
    \end{itemize}
\end{frame}

\begin{frame}{Diagnostik Multikolinearitas Lokal}
    \centering
    \includegraphics[width=0.55\textwidth]{figs/map_condition_number.png} \\
    \vspace{0.1cm}
    \textbf{Condition Number < 30} di seluruh wilayah. \\
    Model GWR stabil dan tidak mengalami masalah multikolinearitas lokal.
\end{frame}

\section{Kesimpulan}
\begin{frame}{Kesimpulan}
    \begin{enumerate}
        \item \textbf{Standardisasi Data}: Penting untuk membandingkan pengaruh relatif variabel ($X_1, X_2, X_3$).
        \item \textbf{Superioritas GWR}: Kernel \textbf{Adaptive Bisquare} meningkatkan $R^2$ dari 0.038 (OLS) menjadi 0.204.
        \item \textbf{Heterogenitas Spasial}:
        \begin{itemize}
            \item \textbf{IMD}: Konsisten berpengaruh positif signifikan.
            \item \textbf{Illness \& Crowded}: Pengaruhnya bersifat lokal (hanya signifikan di beberapa area).
        \end{itemize}
        \item \textbf{Implikasi}: Kebijakan kesehatan harus mempertimbangkan karakteristik lokal masing-masing wilayah.
    \end{enumerate}
\end{frame}

\begin{frame}[allowframebreaks]{Daftar Pustaka}
    \bibliographystyle{apalike}
    %\bibliography{references} % Jika pakai bibtex, tapi kita manual aja sesuai laporan
    \begin{thebibliography}{9}
        \bibitem{gdsl}
        Rey, S. J., Arribas-Bel, D., \& Wolf, L. J. (2020). \textit{Geographic Data Science with Python}.
        \bibitem{fotheringham2002}
        Fotheringham, A. S., et al. (2002). \textit{Geographically Weighted Regression}. Wiley.
        \bibitem{mgwr2019}
        Oshan, T. M., et al. (2019). mgwr: A Python implementation of multiscale GWR. \textit{IJGI}.
    \end{thebibliography}
\end{frame}

\begin{frame}
    \centering
    \Huge \textbf{Terima Kasih}
\end{frame}

\end{document}
