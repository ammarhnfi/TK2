\documentclass{beamer}
\usetheme{Madrid}
\usecolortheme{default}

\usepackage{graphicx}
\usepackage{booktabs}
%\usepackage[indonesian]{babel} % Disabled to prevent PGF conflict

\title[Analisis GWR COVID-19 Inggris]{ANALISIS GEOGRAPHICALLY WEIGHTED REGRESSION (GWR) PADA KASUS COVID-19 DI INGGRIS}
\subtitle{MAKALAH}
\author[Ammar, Norman, Kirono]{
    Ammar Hanafi (2206051582) \texorpdfstring{\\}{, }
    Norman Mowlana Aziz (2206025470) \texorpdfstring{\\}{, }
    Kirono Dwi Saputro (2106656365)
}
\institute[Univ. Indonesia]{
    Program Studi Sarjana Statistika \\
    Fakultas Matematika dan Ilmu Pengetahuan Alam \\
    Universitas Indonesia
}
\date{Juni 2026}

\logo{\includegraphics[height=1cm]{figs/makara.png}}

\begin{document}

\frame{\titlepage}

\begin{frame}{Daftar Isi}
    \tableofcontents
\end{frame}

\section{Pendahuluan}
\begin{frame}{Latar Belakang dan Tujuan}
    \textbf{Latar Belakang}
    \begin{itemize}
        \item COVID-19 memiliki dampak spasial yang tidak seragam.
        \item Faktor risiko (etnis, kesehatan) mungkin memiliki pengaruh yang berbeda di setiap wilayah (Heterogenitas Spasial).
        \item Regresi Global (OLS) mengasumsikan parameter konstan, yang mungkin bias.
    \end{itemize}
    \vspace{0.5cm}
    \textbf{Tujuan}
    \begin{itemize}
        \item Memodelkan tingkat kasus COVID-19 di Inggris.
        \item Menganalisis pengaruh Etnis Non-Kulit Putih dan Penyakit Jangka Panjang.
        \item Membandingkan model OLS dan GWR.
        \item Memetakan variasi koefisien lokal.
    \end{itemize}
\end{frame}

\section{Metodologi}
\begin{frame}{Data dan Variabel}
    \textbf{Unit Analisis}: Upper Tier Local Authorities (n=149) di Inggris.
    \vspace{0.5cm}
    \begin{table}
        \centering
        \begin{tabular}{ll}
            \toprule
            \textbf{Variabel} & \textbf{Deskripsi} \\
            \midrule
            $Y$ (Dependen) & Rate COVID-19 per 100k penduduk \\
            $X_1$ (Independen) & Proporsi Etnis Non-Kulit Putih \\
            $X_2$ (Independen) & Proporsi Penyakit Jangka Panjang \\
            \bottomrule
        \end{tabular}
    \end{table}
\end{frame}

\begin{frame}{Metode Analisis}
    \begin{enumerate}
        \item \textbf{Analisis Deskriptif}: Eksplorasi data dan peta sebaran.
        \item \textbf{Global OLS}: $Y = \beta_0 + \beta_1 X_1 + \beta_2 X_2 + \epsilon$
        \item \textbf{Uji Asumsi}: Breusch-Pagan Test untuk heteroskedastisitas.
        \item \textbf{GWR (Geographically Weighted Regression)}:
        \[ y_i = \beta_0(u_i,v_i) + \sum \beta_k(u_i,v_i) x_{ik} + \epsilon_i \]
        Menggunakan kernel \textit{Adaptive Bisquare}.
        \item \textbf{Diagnostik}: Local Condition Number (Multikolinearitas).
    \end{enumerate}
\end{frame}

\section{Hasil dan Pembahasan}
\begin{frame}{Eksplorasi Data}
    \begin{columns}
        \column{0.5\textwidth}
        \centering
        \includegraphics[width=0.9\textwidth]{figs/map_covid19_r.png} \\
        \tiny Sebaran Kasus COVID-19
        \column{0.5\textwidth}
        \centering
        \includegraphics[width=0.9\textwidth]{figs/map_ethnic.png} \\
        \tiny Sebaran Etnis Non-Kulit Putih
    \end{columns}
    \vspace{0.3cm}
    \centering
    \tiny Pola spasial terlihat mengelompok (cluster), mengindikasikan autokorelasi spasial.
\end{frame}

\begin{frame}{Hasil Model Global (OLS)}
    \begin{table}
        \centering
        \tiny
        
\begin{tabular}{lclc}
\toprule
\textbf{Dep. Variable:}    &    covid19\_r    & \textbf{  R-squared:         } &     0.019   \\
\textbf{Model:}            &       OLS        & \textbf{  Adj. R-squared:    } &     0.006   \\
\textbf{Method:}           &  Least Squares   & \textbf{  F-statistic:       } &     1.432   \\
\textbf{Date:}             & Sun, 08 Feb 2026 & \textbf{  Prob (F-statistic):} &    0.242    \\
\textbf{Time:}             &     11:41:04     & \textbf{  Log-Likelihood:    } &   -404.24   \\
\textbf{No. Observations:} &         149      & \textbf{  AIC:               } &     814.5   \\
\textbf{Df Residuals:}     &         146      & \textbf{  BIC:               } &     823.5   \\
\textbf{Df Model:}         &           2      & \textbf{                     } &             \\
\textbf{Covariance Type:}  &    nonrobust     & \textbf{                     } &             \\
\bottomrule
\end{tabular}
\begin{tabular}{lcccccc}
                     & \textbf{coef} & \textbf{std err} & \textbf{t} & \textbf{P$> |$t$|$} & \textbf{[0.025} & \textbf{0.975]}  \\
\midrule
\textbf{const}       &       4.3646  &        2.209     &     1.975  &         0.050        &       -0.002    &        8.731     \\
\textbf{ethnic}      &       3.7597  &        2.222     &     1.692  &         0.093        &       -0.631    &        8.150     \\
\textbf{lt\_illness} &      10.6082  &       11.128     &     0.953  &         0.342        &      -11.385    &       32.602     \\
\bottomrule
\end{tabular}
\begin{tabular}{lclc}
\textbf{Omnibus:}       & 39.290 & \textbf{  Durbin-Watson:     } &    1.687  \\
\textbf{Prob(Omnibus):} &  0.000 & \textbf{  Jarque-Bera (JB):  } &   76.114  \\
\textbf{Skew:}          &  1.192 & \textbf{  Prob(JB):          } & 2.97e-17  \\
\textbf{Kurtosis:}      &  5.565 & \textbf{  Cond. No.          } &     38.9  \\
\bottomrule
\end{tabular}
%\caption{OLS Regression Results}


Notes: Standard Errors assume that the covariance matrix of the errors is correctly specified.
    \end{table}
    \begin{itemize}
        \item $R^2$ sangat rendah (0.019).
        \item Hanya variabel Etnis yang signifikan pada taraf nyata 10\%.
        \item Model global tidak mampu menjelaskan variasi data dengan baik.
    \end{itemize}
\end{frame}

\begin{frame}{Bukti Heterogenitas Spasial}
    \begin{columns}
        \column{0.5\textwidth}
        \textbf{Uji Breusch-Pagan}
        \begin{itemize}
            \item p-value = 0.8739 ($> 0.05$)
            \item Secara statistik gagal tolak H0.
            \item NAMUN... visualisasi residual berkata lain.
        \end{itemize}
        \column{0.5\textwidth}
        \centering
        \includegraphics[width=0.9\textwidth]{figs/map_ols_residuals.png} \\
        \tiny Peta Residual OLS (Clustered)
    \end{columns}
\end{frame}

\begin{frame}{Hasil Model GWR}
    \begin{table}
        \centering
        \begin{tabular}{lr}
\toprule
 & Value \\
Metric &  \\
\midrule
Bandwidth & 113.0000 \\
AICc & 793.3382 \\
R2 & 0.2170 \\
Adj. R2 & 0.1745 \\
\bottomrule
\end{tabular}

    \end{table}
    \begin{itemize}
        \item \textbf{AICc Turun}: 814.5 $\rightarrow$ 793.3 (Model Membaik).
        \item \textbf{$R^2$ Naik}: 0.019 $\rightarrow$ 0.221.
        \item GWR berhasil menangkap variasi lokal yang dilewatkan OLS.
    \end{itemize}
\end{frame}

\begin{frame}{Peta Koefisien Lokal: Etnis}
    \begin{columns}
        \column{0.5\textwidth}
        \centering
        \includegraphics[width=1\textwidth]{figs/ethnic_sig.png}
        \column{0.5\textwidth}
        \textbf{Interpretasi}:
        \begin{itemize}
            \item Area Abu-abu: Tidak signifikan ($|t| < 1.96$).
            \item Area Berwarna: Signifikan.
            \item Pengaruh etnis positif kuat di wilayah tertentu (Kuning/Hijau).
        \end{itemize}
    \end{columns}
\end{frame}

\begin{frame}{Peta Koefisien Lokal: Penyakit Jangka Panjang}
    \begin{columns}
        \column{0.5\textwidth}
        \centering
        \includegraphics[width=1\textwidth]{figs/illness_sig.png}
        \column{0.5\textwidth}
        \textbf{Interpretasi}:
        \begin{itemize}
            \item Variabel ini tidak signifikan di model OLS global, TAPI signifikan positif di beberapa wilayah lokal (GWR).
            \item Menunjukkan pentingnya analisis lokal untuk mengungkap hubungan tersembunyi.
        \end{itemize}
    \end{columns}
\end{frame}

\begin{frame}{Diagnostik Multikolinearitas Lokal}
    \centering
    \includegraphics[width=0.55\textwidth]{figs/map_condition_number.png} \\
    \vspace{0.1cm}
    \textbf{Condition Number < 30} di seluruh wilayah. \\
    Model GWR stabil dan tidak mengalami masalah multikolinearitas lokal.
\end{frame}

\section{Kesimpulan}
\begin{frame}{Kesimpulan}
    \begin{enumerate}
        \item Model GWR jauh lebih \textbf{superior} dibandingkan OLS (AICc lebih rendah, $R^2$ lebih tinggi).
        \item Terdapat \textbf{heterogenitas spasial} yang signifikan dalam pengaruh Etnis dan Kesehatan terhadap COVID-19.
        \item Kebijakan penanganan COVID-19 harus \textbf{spesifik wilayah}, tidak bisa disamaratakan secara nasional.
    \end{enumerate}
\end{frame}

\begin{frame}[allowframebreaks]{Daftar Pustaka}
    \bibliographystyle{apalike}
    %\bibliography{references} % Jika pakai bibtex, tapi kita manual aja sesuai laporan
    \begin{thebibliography}{9}
        \bibitem{gdsl}
        Rey, S. J., Arribas-Bel, D., \& Wolf, L. J. (2020). \textit{Geographic Data Science with Python}.
        \bibitem{fotheringham2002}
        Fotheringham, A. S., et al. (2002). \textit{Geographically Weighted Regression}. Wiley.
        \bibitem{mgwr2019}
        Oshan, T. M., et al. (2019). mgwr: A Python implementation of multiscale GWR. \textit{IJGI}.
    \end{thebibliography}
\end{frame}

\begin{frame}
    \centering
    \Huge \textbf{Terima Kasih}
\end{frame}

\end{document}
