\documentclass[12pt,a4paper]{article}
\usepackage{graphicx}
\usepackage{geometry}
\usepackage{hyperref}
\usepackage[utf8]{inputenc}
\usepackage[indonesian]{babel}
\usepackage{float}
\usepackage{booktabs}
\usepackage{amsmath}
\usepackage{subcaption}

\geometry{
	left=3cm,
	right=2.5cm,
	top=3cm,
	bottom=2.5cm,
}

\begin{document}

\begin{titlepage}
    \centering
    \includegraphics[width=3.5cm]{figs/makara.png} \par\vspace{1cm}
    {\normalsize \textbf{UNIVERSITAS INDONESIA}} \par\vspace{3cm}

    {\normalsize \textbf{ANALISIS GEOGRAPHICALLY WEIGHTED REGRESSION (GWR) PADA KASUS COVID-19 DI INGGRIS}} \par\vspace{4cm}

    {\normalsize \textbf{MAKALAH}} \par\vspace{3cm}

    {\normalsize \textbf{AMMAR HANAFI (2206051582)}} \par
    {\normalsize \textbf{NORMAN MOWLANA AZIZ (2206025470)}} \par
    {\normalsize \textbf{KIRONO DWI SAPUTRO (2106656365)}} \par
    {\normalsize \textbf{DEVANA SOLEA (2306262402)}} \par\vfill

    {\normalsize \textbf{FAKULTAS MATEMATIKA DAN ILMU PENGETAHUAN ALAM}} \par
    {\normalsize \textbf{PROGRAM STUDI SARJANA STATISTIKA}} \par
    {\normalsize \textbf{DEPOK}} \par
    {\normalsize \textbf{JUNI 2026}}
\end{titlepage}

\tableofcontents
\newpage
\listoftables
\newpage
\listoffigures
\newpage
%===================================
% BAB 1 - PENDAHULUAN
%===================================
\newpage
\begin{center}
    \bfseries\large
    BAB I \\
    PENDAHULUAN
\end{center}
\addcontentsline{toc}{section}{BAB I PENDAHULUAN}
\setcounter{section}{1}
\setcounter{subsection}{0}

\subsection{Latar Belakang}
Pandemi COVID-19 telah memberikan dampak yang signifikan di seluruh dunia, termasuk di Inggris. Penyebaran virus ini tidak merata secara geografis, dipengaruhi oleh berbagai faktor sosial-ekonomi dan demografis. Memahami variasi spasial dalam faktor-faktor risiko ini sangat penting untuk perumusan kebijakan kesehatan masyarakat yang efektif. Metode regresi konvensional (OLS) seringkali mengabaikan aspek heterogenitas spasial ini, sehingga diperlukan pendekatan Geographically Weighted Regression (GWR).

\subsection{Rumusan Masalah}
\begin{enumerate}
    \item Apakah terdapat heterogenitas spasial dalam hubungan antara faktor etnisitas dan kesehatan jangka panjang terhadap tingkat kasus COVID-19 di Inggris?
    \item Bagaimana variasi lokal dari pengaruh faktor-faktor tersebut di berbagai wilayah?
\end{enumerate}

\subsection{Tujuan}
Tujuan dari analisis ini adalah untuk memodelkan hubungan spasial antara tingkat kasus COVID-19 dengan tingkat deprivasi (IMD), kesehatan jangka panjang, dan kepadatan hunian, serta mengidentifikasi variasi lokal dari hubungan tersebut menggunakan GWR.

Selanjutnya, bab berikut akan membahas landasan teori yang mendasari metode analisis yang digunakan dalam penelitian ini.

%===================================
% BAB 2 - LANDASAN TEORI
%===================================
\newpage
\begin{center}
    \bfseries\large
    BAB II \\
    LANDASAN TEORI
\end{center}
\addcontentsline{toc}{section}{BAB II LANDASAN TEORI}
\setcounter{section}{2}
\setcounter{subsection}{0}

\subsection{Regresi Linear Berganda (OLS)}
Regresi linear berganda adalah metode statistik untuk memodelkan hubungan antara satu variabel dependen dengan dua atau lebih variabel independen. Model OLS mengasumsikan bahwa parameter regresi bersifat konstan di seluruh wilayah pengamatan (stasioneritas spasial). Persamaan umumnya adalah:
\[ Y = \beta_0 + \sum_{k=1}^{p} \beta_k X_k + \epsilon \]

dengan:
\begin{itemize}
    \item $Y$: Variabel dependen (respon).
    \item $\beta_0$: Intersep atau konstanta model global.
    \item $\beta_k$: Koefisien regresi global untuk variabel independen ke-$k$.
    \item $X_k$: Variabel independen (prediktor) ke-$k$.
    \item $\epsilon$: Galat (error) yang diasumsikan berdistribusi normal dengan mean 0 dan varians konstan ($\sigma^2$).
\end{itemize}

Model OLS ini menjadi dasar perbandingan untuk mengevaluasi efektivitas model spasial yang akan dibahas selanjutnya.

\subsection{Uji Asumsi Klasik: Heterogenitas Spasial}
Salah satu asumsi penting dalam OLS adalah homoskedastisitas, yaitu varians error yang konstan. Pelanggaran terhadap asumsi ini (heteroskedastisitas) seringkali mengindikasikan adanya heterogenitas spasial. Uji Breusch-Pagan dapat digunakan untuk mendeteksi adanya heteroskedastisitas ini.

\subsection{Geographically Weighted Regression (GWR)}
GWR adalah pengembangan dari regresi linear yang memungkinkan parameter model bervariasi secara lokal. Hal ini dilakukan dengan memberikan bobot pada observasi di sekitar lokasi estimasi. Model GWR didefinisikan sebagai:
\[ y_i = \beta_0(u_i, v_i) + \sum_{k} \beta_k(u_i, v_i) x_{ik} + \epsilon_i \]

dengan:
\begin{itemize}
    \item $y_i$: Nilai observasi variabel dependen di lokasi $i$.
    \item $(u_i, v_i)$: Koordinat geografis (longitude, latitude) dari lokasi $i$.
    \item $\beta_0(u_i, v_i)$: Intersep model yang bervariasi secara lokal di lokasi $i$.
    \item $\beta_k(u_i, v_i)$: Koefisien regresi lokal untuk variabel independen ke-$k$ di lokasi $i$.
    \item $x_{ik}$: Nilai variabel independen ke-$k$ di lokasi $i$.
    \item $\epsilon_i$: Galat di lokasi $i$.
\end{itemize}

di mana $(u_i, v_i)$ adalah koordinat lokasi ke-$i$. Pembobotan dilakukan menggunakan fungsi kernel yang memberikan bobot lebih besar untuk lokasi yang berdekatan. Dalam penelitian ini, digunakan dua jenis fungsi kernel adaptif:

\begin{itemize}
    \item \textbf{Adaptive Gaussian}:
    \[ w_{ij} = \exp \left( - \frac{d_{ij}^2}{2\theta_i^2} \right) \]
    \item \textbf{Adaptive Bisquare}:
    \[ w_{ij} = \begin{cases} 
    \left( 1 - \left( \frac{d_{ij}}{b_i} \right)^2 \right)^2 & \text{jika } d_{ij} < b_i \\ 
    0 & \text{jika } d_{ij} \geq b_i 
    \end{cases} \]
\end{itemize}

di mana $d_{ij}$ adalah jarak antara lokasi $i$ dan $j$, serta $b_i$ (atau $\theta_i$) adalah bandwidth adaptif yang menentukan jumlah tetangga terdekat (k-nearest neighbors) yang digunakan dalam estimasi. Metode ini memungkinkan kita untuk menangkap fenomena heterogenitas spasial secara lebih detail.

Setelah memahami landasan teori, bab selanjutnya akan menguraikan metodologi penelitian yang diterapkan, mulai dari data hingga tahapan analisis.

%===================================
% BAB 3 - METODE PENELITIAN
%===================================
\newpage
\begin{center}
    \bfseries\large
    BAB III \\
    METODE PENELITIAN
\end{center}
\addcontentsline{toc}{section}{BAB III METODE PENELITIAN}
\setcounter{section}{3}
\setcounter{subsection}{0}

\subsection{Data}
Analisis ini menggunakan dataset kasus COVID-19 di Inggris (level Upper Tier Local Authority).
\begin{itemize}
    \item \textbf{Variabel Dependen ($Y$)}: Tingkat kasus COVID-19 per 100.000 penduduk (\texttt{covid19\_r}).
    \item \textbf{Variabel Independen ($X_1$)}: Index of Multiple Deprivation (\texttt{imd}).
    \item \textbf{Variabel Independen ($X_2$)}: Proporsi penduduk dengan penyakit jangka panjang (\texttt{lt\_illness}).
    \item \textbf{Variabel Independen ($X_3$)}: Proporsi rumah tangga yang padat (\texttt{crowded}).
\end{itemize}

Sebelum dilakukan analisis regresi (OLS dan GWR), seluruh variabel (dependen dan independen) dilakukan transformasi \textit{Z-score standardization} (mean=0, std=1). Hal ini bertujuan untuk menyetarakan skala antar variabel sehingga nilai koefisien yang dihasilkan dapat diperbandingkan secara langsung (\textit{comparable}) dan meminimalkan masalah numerik dalam estimasi model.

Contoh data (10 baris pertama) yang digunakan dalam analisis, mencakup informasi lokasi dan variabel penelitian, dapat dilihat pada Tabel \ref{tab:data_sample}.

\begin{table}[H]
    \centering
    \caption{Cuplikan Dataset Analisis (Lokasi dan Variabel)}
    \label{tab:data_sample}
    \tiny
    \begin{tabular}{llrrrrrr}
\toprule
 & Region Name & Longitude & Latitude & Covid Rate & IMD & Illness & Crowded \\
\midrule
0 & Hartlepool & -1.2702 & 54.6761 & 8.6930 & 35.0370 & 0.2316 & 0.0094 \\
1 & Middlesbrough & -1.2110 & 54.5447 & 14.4496 & 40.4600 & 0.2086 & 0.0172 \\
2 & Redcar and Cleveland & -1.0061 & 54.5675 & 3.6989 & 29.7920 & 0.2272 & 0.0078 \\
3 & Stockton-on-Tees & -1.3066 & 54.5569 & 7.8284 & 25.7900 & 0.1900 & 0.0093 \\
4 & Darlington & -1.5683 & 54.5353 & 6.6310 & 25.6570 & 0.1963 & 0.0085 \\
5 & Halton & -2.6885 & 53.3342 & 7.9525 & 32.3250 & 0.2136 & 0.0098 \\
6 & Warrington & -2.5617 & 53.3916 & 14.3402 & 18.9420 & 0.1731 & 0.0085 \\
7 & Blackburn with Darwen & -2.4636 & 53.7008 & 10.8483 & 36.0130 & 0.2023 & 0.0262 \\
8 & Blackpool & -3.0220 & 53.8216 & 5.6312 & 45.0390 & 0.2557 & 0.0094 \\
9 & Kingston upon Hull, City of & -0.3038 & 53.7692 & 10.9202 & 40.5640 & 0.1965 & 0.0163 \\
\bottomrule
\end{tabular}

\end{table}

Peta wilayah studi dengan batas administrasi dan nama lokasi (sebagian) ditampilkan pada Gambar \ref{fig:map_boundaries}. Peta ini dilengkapi dengan grid garis lintang (latitude) dan bujur (longitude) untuk memudahkan identifikasi posisi geografis.

\begin{figure}[H]
    \centering
    \includegraphics[width=0.9\textwidth]{figs/map_boundaries.png}
    \caption{Peta Wilayah Studi dengan Batas Administrasi dan Koordinat}
    \label{fig:map_boundaries}
\end{figure}

\subsection{Tahapan Analisis}
\begin{enumerate}
    \item Eksplorasi data (Statistika Deskriptif).
    \item Pemodelan Regresi Global (OLS).
    \item Uji asumsi heterogenitas spasial (Uji Breusch-Pagan).
    \item Pemodelan GWR dengan \textit{adaptive kernel}.
    \item Interpretasi dan visualisasi hasil.
\end{enumerate}

Dengan mengikuti tahapan tersebut, diharapkan dapat diperoleh pemahaman yang komprehensif mengenai pola spasial COVID-19. Hasil dari analisis ini akan dipaparkan secara rinci pada bab berikut.

%===================================
% BAB 4 - ANALISIS DAN PEMBAHASAN
%===================================
\newpage
\begin{center}
    \bfseries\large
    BAB IV \\
    ANALISIS DAN PEMBAHASAN
\end{center}
\addcontentsline{toc}{section}{BAB IV ANALISIS DAN PEMBAHASAN}
\setcounter{section}{4}
\setcounter{subsection}{0}

\subsection{Statistik Deskriptif}
\begin{table}[H]
    \centering
    \caption{Statistik Deskriptif Variabel}
    \label{tab:desc_stats}
    \begin{tabular}{lrrrrrrrrrr}
\toprule
 & count & mean & std & min & 25\% & 50\% & 75\% & max & skew & kurtosis \\
\midrule
covid19\_r & 149.0000 & 6.8579 & 3.6956 & 0.0000 & 4.1015 & 6.2406 & 8.6930 & 23.7640 & 1.1430 & 2.4236 \\
ethnic & 149.0000 & 0.1649 & 0.1635 & 0.0146 & 0.0403 & 0.1001 & 0.2552 & 0.7103 & 1.2440 & 0.6025 \\
lt\_illness & 149.0000 & 0.1766 & 0.0326 & 0.1120 & 0.1497 & 0.1756 & 0.2027 & 0.2557 & 0.1504 & -0.8086 \\
\bottomrule
\end{tabular}

\end{table}

Berdasarkan Tabel \ref{tab:desc_stats}, variabel tingkat kasus COVID-19 memiliki variasi antar wilayah. Variabel IMD (\texttt{imd}) mengukur tingkat deprivasi relatif, di mana nilai yang lebih tinggi menunjukkan tingkat deprivasi yang lebih tinggi. Proporsi penduduk dengan penyakit jangka panjang (\texttt{lt\_illness}) dan kepadatan hunian (\texttt{crowded}) juga menunjukkan variasi spasial yang perlu diperhatikan.

Gambar \ref{fig:boxplot} menampilkan distribusi data variabel melalui boxplot. Terlihat adanya variasi nilai antar wilayah.

\begin{figure}[H]
    \centering
    \includegraphics[width=0.8\textwidth]{figs/boxplot.png}
    \caption{Boxplot Variabel Penelitian}
    \label{fig:boxplot}
\end{figure}

Boxplot pada Gambar \ref{fig:boxplot} menvisualisasikan sebaran data untuk semua variabel. Pencilan (outlier) mungkin teramati pada beberapa variabel, mengindikasikan adanya wilayah-wilayah dengan karakteristik ekstrem.

Investigasi lebih lanjut mengenai hubungan antar variabel dilakukan menggunakan analisis korelasi (Gambar \ref{fig:pairplot} dan \ref{fig:heatmap}).

\begin{figure}[H]
    \centering
    \includegraphics[width=0.9\textwidth]{figs/pairplot.png}
    \caption{Pairplot Variabel Penelitian}
    \label{fig:pairplot}
\end{figure}

Gambar \ref{fig:pairplot} menampilkan plot pasangan antar variabel. Diagonal utama menunjukkan distribusi masing-masing variabel, sedangkan scatter plot di luar diagonal memperlihatkan pola hubungan bivariat.

\begin{figure}[H]
    \centering
    \includegraphics[width=0.7\textwidth]{figs/heatmap.png}
    \caption{Matriks Korelasi (Heatmap)}
    \label{fig:heatmap}
\end{figure}

Matriks korelasi pada Gambar \ref{fig:heatmap} memberikan kuantifikasi hubungan linear tersebut. Nilai korelasi yang tinggi (mendekati 1 atau -1) mengindikasikan hubungan yang kuat, sementara nilai mendekati 0 menunjukkan hubungan yang lemah. Dalam konteks ini, kita dapat melihat sejauh mana prediktor (IMD, Illness, Crowded) berkorelasi dengan respons (COVID-19) serta potensi multikolinearitas antar prediktor.

\begin{figure}[H]
    \centering
    \includegraphics[width=0.6\textwidth]{figs/correlation_matrix.png}
    \caption{Matriks Korelasi Pearson}
    \label{fig:corr}
\end{figure}

Gambar \ref{fig:corr} menunjukkan koefisien korelasi antar variabel yang digunakan.

Peta sebaran variabel dependen dan independen disajikan pada gambar berikut.

\begin{figure}[H]
    \centering
    \includegraphics[width=0.45\textwidth]{figs/map_covid19_r.png}
    \includegraphics[width=0.45\textwidth]{figs/map_imd.png} \\
    \includegraphics[width=0.45\textwidth]{figs/map_lt_illness.png}
    \includegraphics[width=0.45\textwidth]{figs/map_crowded.png}
    \caption{Peta Sebaran Variabel (Atas: COVID-19 \& IMD, Bawah: Penyakit \& Crowded)}
    \label{fig:maps_vars}
\end{figure}

Peta sebaran (Gambar \ref{fig:maps_vars}) memperlihatkan pola spasial yang menarik. Tingkat kasus COVID-19 yang tinggi cenderung terkonsentrasi di beberapa wilayah tertentu. Pola ini tampaknya memiliki kemiripan visual dengan sebaran proporsi etnis di beberapa area, namun hubungannya perlu diuji lebih lanjut melalui pemodelan statistik.

\subsection{Hasil Pemodelan Global (OLS)}
Tabel berikut menunjukkan ringkasan hasil estimasi model OLS global.

\begin{table}[H]
    \centering
    \caption{Ringkasan Model OLS Global}
    \label{tab:ols_summary}
    {\footnotesize
        
\begin{tabular}{lclc}
\toprule
\textbf{Dep. Variable:}    &    covid19\_r    & \textbf{  R-squared:         } &     0.019   \\
\textbf{Model:}            &       OLS        & \textbf{  Adj. R-squared:    } &     0.006   \\
\textbf{Method:}           &  Least Squares   & \textbf{  F-statistic:       } &     1.432   \\
\textbf{Date:}             & Sun, 08 Feb 2026 & \textbf{  Prob (F-statistic):} &    0.242    \\
\textbf{Time:}             &     11:41:04     & \textbf{  Log-Likelihood:    } &   -404.24   \\
\textbf{No. Observations:} &         149      & \textbf{  AIC:               } &     814.5   \\
\textbf{Df Residuals:}     &         146      & \textbf{  BIC:               } &     823.5   \\
\textbf{Df Model:}         &           2      & \textbf{                     } &             \\
\textbf{Covariance Type:}  &    nonrobust     & \textbf{                     } &             \\
\bottomrule
\end{tabular}
\begin{tabular}{lcccccc}
                     & \textbf{coef} & \textbf{std err} & \textbf{t} & \textbf{P$> |$t$|$} & \textbf{[0.025} & \textbf{0.975]}  \\
\midrule
\textbf{const}       &       4.3646  &        2.209     &     1.975  &         0.050        &       -0.002    &        8.731     \\
\textbf{ethnic}      &       3.7597  &        2.222     &     1.692  &         0.093        &       -0.631    &        8.150     \\
\textbf{lt\_illness} &      10.6082  &       11.128     &     0.953  &         0.342        &      -11.385    &       32.602     \\
\bottomrule
\end{tabular}
\begin{tabular}{lclc}
\textbf{Omnibus:}       & 39.290 & \textbf{  Durbin-Watson:     } &    1.687  \\
\textbf{Prob(Omnibus):} &  0.000 & \textbf{  Jarque-Bera (JB):  } &   76.114  \\
\textbf{Skew:}          &  1.192 & \textbf{  Prob(JB):          } & 2.97e-17  \\
\textbf{Kurtosis:}      &  5.565 & \textbf{  Cond. No.          } &     38.9  \\
\bottomrule
\end{tabular}
%\caption{OLS Regression Results}


Notes: Standard Errors assume that the covariance matrix of the errors is correctly specified.
    }
\end{table}

Hasil estimasi OLS pada Tabel \ref{tab:ols_summary} menunjukkan nilai $R^2$ sebesar 0.038, yang berarti model global hanya mampu menjelaskan 3.8\% variasi tingkat kasus COVID-19. Secara parsial, hanya variabel IMD yang berpengaruh signifikan pada taraf nyata 5\% dengan koefisien 0.1393 ($p=0.026$). Variabel penyakit jangka panjang dan kepadatan hunian tidak signifikan secara statistik ($p > 0.05$). Rendahnya performa model ini dan signifikansi yang terbatas mengindikasikan fluktuasi lokal yang tidak tertangkap oleh model global.

\subsection{Uji Asumsi Heterogenitas Spasial}
\begin{table}[H]
    \centering
    \caption{Hasil Uji Breusch-Pagan}
    \label{tab:bp_test}
    \begin{tabular}{lr}
\toprule
 & Value \\
\midrule
LM Statistic & 1.5498 \\
LM-Test p-value & 0.6708 \\
F-Statistic & 0.5080 \\
F-Test p-value & 0.6774 \\
\bottomrule
\end{tabular}

\end{table}

Hasil uji Breusch-Pagan (Tabel \ref{tab:bp_test}) memberikan nilai p-value yang cukup besar (0.6708). Meskipun secara statistik gagal menolak homoskedastisitas, visualisasi residual spasial seringkali dapat mengungkapkan pola yang tersembunyi.

\begin{figure}[H]
    \centering
    \includegraphics[width=0.8\textwidth]{figs/map_ols_residuals.png}
    \caption{Peta Residual Model OLS}
    \label{fig:ols_resid}
\end{figure}

Peta residual OLS (Gambar \ref{fig:ols_resid}) memperlihatkan adanya kecenderungan pengelompokan (clustering) nilai residual positif dan negatif di wilayah-wilayah tertentu. Pola visual ini mengindikasikan potensi autokorelasi spasial yang tidak sepenuhnya tertangkap oleh uji global, sehingga pemodelan lokal (GWR) tetap relevan untuk dilakukan.

\subsection{Hasil Pemodelan GWR}
Untuk mendapatkan model terbaik, dilakukan perbandingan antara dua fungsi kernel adaptif: Bisquare dan Gaussian. Ringkasan perbandingan performa model disajikan pada Tabel \ref{tab:model_comparison}.

\begin{table}[H]
    \centering
    \caption{Perbandingan Model OLS dan GWR (Bisquare vs Gaussian)}
    \label{tab:model_comparison}
    \begin{tabular}{lrrr}
\toprule
 & AICc & R2 & Adj. R2 \\
Model &  &  &  \\
\midrule
OLS & 424.1189 & 0.0377 & 0.0178 \\
GWR Bisquare & 407.4528 & 0.2044 & 0.1579 \\
GWR Gaussian & 408.6839 & 0.1967 & 0.1503 \\
\bottomrule
\end{tabular}

\end{table}

Berdasarkan Tabel \ref{tab:model_comparison}, model GWR dengan kernel \textbf{Adaptive Bisquare} memberikan performa terbaik dengan nilai AICc terendah (796.98) dan $R^2$ tertinggi (0.2044), dibandingkan dengan OLS ($R^2=0.0377$) maupun GWR Gaussian ($R^2=0.1967$). Oleh karena itu, interpretasi selanjutnya akan didasarkan pada model GWR Bisquare.

Analisis lebih lanjut mengenai kesesuaian model dapat dilihat dari peta residual pada Gambar \ref{fig:resid_comp}.

\begin{figure}[H]
    \centering
    \includegraphics[width=1.0\textwidth]{figs/residual_comparison.png}
    \caption{Perbandingan Peta Residual: OLS vs GWR Bisquare vs GWR Gaussian}
    \label{fig:resid_comp}
\end{figure}

Gambar di atas menunjukkan bahwa residual dari model GWR (tengah dan kanan) cenderung lebih acak dan memiliki variasi yang lebih kecil dibandingkan OLS (kiri), yang mengindikasikan bahwa model GWR berhasil menangkap heterogenitas spasial dengan lebih baik.

\subsubsection{Variabilitas Koefisien Lokal}
Selain peta spasial, variasi koefisien lokal GWR juga dapat dilihat melalui boxplot pada Gambar \ref{fig:gwr_beta_boxplot}. Rentang nilai (box) dan whisker menunjukkan seberapa bervariasi pengaruh masing-masing variabel prediktor secara lokal. Jika rentang interkuartil sempit, berarti pengaruh variabel tersebut relatif homogen; sebaliknya, jika lebar, menunjukkan adanya heterogenitas lokal yang kuat.

\begin{figure}[H]
    \centering
    \includegraphics[width=0.8\textwidth]{figs/gwr_beta_boxplot.png}
    \caption{Boxplot Distribusi Koefisien Lokal (Beta) GWR}
    \label{fig:gwr_beta_boxplot}
\end{figure}

\subsubsection{Kebaikan Model Lokal (\texorpdfstring{$R^2$}{R-squared})}
Variasi lokal nilai $R^2$ (Gambar \ref{fig:local_r2}) menunjukkan bahwa kemampuan model dalam menjelaskan variasi kasus COVID-19 berbeda-beda di setiap wilayah.

\begin{figure}[H]
    \centering
    \includegraphics[width=0.8\textwidth]{figs/covid_gwr_r2.png}
    \caption{Peta Local $R^2$}
    \label{fig:local_r2}
\end{figure}

Peta Local $R^2$ di atas memperlihatkan bahwa di beberapa wilayah, model GWR memiliki daya jel yang tinggi (warna lebih terang/kuning), sementara di wilayah lain performanya lebih rendah (warna lebih gelap). Hal ini wajar dalam analisis spasial, di mana faktor-faktor penjelas mungkin memiliki kekuatan prediksi yang berbeda tergantung pada konteks lokal masing-masing daerah.

\subsubsection{Interpretasi Koefisien Spasial}
Berikut adalah peta estimasi parameter lokal ($\beta$) dan nilai t-statistik untuk setiap variabel prediktor.

\paragraph{Variabel IMD (Deprivasi)}
Gambar \ref{fig:coef_imd} menunjukkan sebaran koefisien lokal dan nilai p-value untuk variabel IMD.

\begin{figure}[H]
    \centering
    \begin{subfigure}[b]{0.48\textwidth}
        \centering
        \includegraphics[width=\textwidth]{figs/covid_gwr_imd_coef.png}
        \caption{Koefisien}
        \label{fig:imd_coef}
    \end{subfigure}
    \hfill
    \begin{subfigure}[b]{0.48\textwidth}
        \centering
        \includegraphics[width=\textwidth]{figs/covid_gwr_imd_pval.png}
        \caption{P-Value}
        \label{fig:imd_pval}
    \end{subfigure}
    \caption{Peta Variabel IMD}
    \label{fig:coef_imd}
\end{figure}

Berdasarkan Gambar \ref{fig:coef_imd}b (P-Value), daerah yang berwarna gelap (ungu/biru tua) memiliki nilai p-value yang kecil, menunjukkan bahwa variabel IMD berpengaruh signifikan secara statistik. Gambar \ref{fig:coef_imd}a (Koefisien) menunjukkan besaran pengaruhnya, di mana koefisien positif (merah) mendominasi area yang signifikan.

\paragraph{Variabel Penyakit Jangka Panjang}
Gambar \ref{fig:coef_illness} menampilkan hasil untuk variabel penyakit jangka panjang.

\begin{figure}[H]
    \centering
    \begin{subfigure}[b]{0.48\textwidth}
        \centering
        \includegraphics[width=\textwidth]{figs/covid_gwr_illness_coef.png}
        \caption{Koefisien}
        \label{fig:illness_coef}
    \end{subfigure}
    \hfill
    \begin{subfigure}[b]{0.48\textwidth}
        \centering
        \includegraphics[width=\textwidth]{figs/covid_gwr_illness_pval.png}
        \caption{P-Value}
        \label{fig:illness_pval}
    \end{subfigure}
    \caption{Peta Variabel Penyakit Jangka Panjang}
    \label{fig:coef_illness}
\end{figure}

Gambar \ref{fig:coef_illness}b memperlihatkan bahwa variabel ini memiliki pengaruh yang signifikan di beberapa bagian wilayah studi. Di wilayah-wilayah tersebut, koefisien cenderung positif.

\paragraph{Variabel Kepadatan Hunian (Crowded)}
\begin{figure}[H]
    \centering
    \begin{subfigure}[b]{0.48\textwidth}
        \centering
        \includegraphics[width=\textwidth]{figs/covid_gwr_crowded_coef.png}
        \caption{Koefisien}
        \label{fig:crowded_coef}
    \end{subfigure}
    \hfill
    \begin{subfigure}[b]{0.48\textwidth}
        \centering
        \includegraphics[width=\textwidth]{figs/covid_gwr_crowded_pval.png}
        \caption{P-Value}
        \label{fig:crowded_pval}
    \end{subfigure}
    \caption{Peta Variabel Crowded}
    \label{fig:coef_crowded}
\end{figure}

Interpretasi visual peta menunjukkan adanya heterogenitas spasial. Wilayah dengan p-value rendah pada Gambar \ref{fig:coef_crowded}b mengidentifikasi area di mana kepadatan hunian menjadi faktor risiko yang nyata. Variasi warna pada Gambar \ref{fig:coef_crowded}a menegaskan bahwa besaran dampak dari kepadatan hunian tidak seragam.

\subsubsection{Diagnostik Multikolinearitas Lokal}
Stabilitas estimasi parameter GWR perlu dipastikan dengan memeriksa multikolinearitas lokal.

\begin{figure}[H]
    \centering
    \includegraphics[width=0.8\textwidth]{figs/map_condition_number.png}
    \caption{Peta Local Condition Number}
    \label{fig:cond_num}
\end{figure}

Gambar \ref{fig:cond_num} menunjukkan sebaran nilai \textit{Condition Number} lokal. Nilai yang rendah mengindikasikan tidak adanya masalah multikolinearitas lokal yang serius antar variabel prediktor (IMD, Penyakit, Crowded) di wilayah studi.

Setelah seluruh tahapan analisis dan diagnostik dilakukan, bagian selanjutnya akan merangkum temuan-temuan utama dalam bentuk kesimpulan.

\subsubsection{Pengelompokan Wilayah Berdasarkan Signifikansi}
Berdasarkan peta t-value (atau p-value), wilayah di Inggris dapat dikelompokkan menjadi dua kategori utama untuk setiap variabel prediktor:
\begin{enumerate}
    \item \textbf{Wilayah Signifikan}: Area di mana variabel prediktor memiliki pengaruh nyata terhadap kasus COVID-19 (p-value $< 0.05$). Pada peta signifikansi, area ini ditandai dengan warna gelap.
    \item \textbf{Wilayah Tidak Signifikan}: Area di mana hubungan antara prediktor dan respon tidak cukup kuat untuk dikatakan berbeda dari nol secara statistik.
\end{enumerate}

Secara spesifik:
\begin{itemize}
    \item \textbf{Klaster Deprivasi (IMD)}: Hampir seluruh wilayah studi menunjukkan signifikansi positif, mengindikasikan bahwa deprivasi adalah faktor risiko universal di seluruh Inggris.
    \item \textbf{Klaster Kesehatan \& Kepadatan}: Signifikansi terbatas pada kantong-kantong wilayah tertentu (spatial pockets). Hal ini menunjukkan bahwa kebijakan intervensi kesehatan dan perumahan harus difokuskan secara spesifik pada klaster-klaster wilayah tersebut, bukan diterapkan secara merata.
\end{itemize}

%===================================
% BAB 5 - KESIMPULAN DAN SARAN
%===================================
\newpage
\begin{center}
    \bfseries\large
    BAB V \\
    KESIMPULAN
\end{center}
\addcontentsline{toc}{section}{BAB V KESIMPULAN}
\setcounter{section}{5}
\setcounter{subsection}{0}
Berdasarkan hasil analisis dan pembahasan yang telah dilakukan, dapat disimpulkan bahwa:
\begin{enumerate}
    \item \textbf{Limitasi Model Global}: Model Regresi Linear Berganda (OLS) kurang efektif dalam menjelaskan variasi tingkat kasus COVID-19 di Inggris, ditunjukkan dengan nilai $R^2$ yang sangat rendah (0.038). Hal ini mengindikasikan bahwa asumsi stasioneritas spasial tidak terpenuhi.
    
    \item \textbf{Keunggulan GWR}: Penerapan Geographically Weighted Regression (GWR) secara signifikan meningkatkan kemampuan prediksi model. Penggunaan kernel \textbf{Adaptive Bisquare} terbukti memberikan hasil terbaik dibandingkan Gaussian, dengan peningkatan $R^2$ menjadi 0.204 dan penurunan AICc dari 813.6 (OLS) menjadi 797.0.
    
    \item \textbf{Heterogenitas Spasial}: Faktor-faktor prediktor memiliki pengaruh yang bervariasi secara lokal:
    \begin{itemize}
        \item \textbf{IMD (Deprivasi)}: Menjadi variabel yang paling konsisten berpengaruh signifikan (p-value lokal $< 0.05$) di banyak wilayah, dengan hubungan positif (semakin tinggi deprivasi, semakin tinggi kasus).
        \item \textbf{Penyakit Jangka Panjang \& Kepadatan Hunian}: Memiliki pengaruh yang lebih terlokalisasi, signifikan hanya di beberapa area tertentu saja.
    \end{itemize}
\end{enumerate}

Penelitian ini menegaskan pentingnya pendekatan spasial lokal dalam epidemiologi untuk mengidentifikasi faktor risiko yang spesifik untuk setiap wilayah, sehingga intervensi kebijakan dapat dilakukan lebih tepat sasaran.

\newpage
\begin{center}
    \section*{DAFTAR PUSTAKA}
\end{center}
\addcontentsline{toc}{section}{DAFTAR PUSTAKA}

\begin{thebibliography}{9}
\renewcommand{\refname}{} % Clear default name to use our centered one
\vspace{-2cm}

\bibitem{gdsl}
Rey, S. J., Arribas-Bel, D., \& Wolf, L. J. (2020). \textit{Geographic Data Science with Python}. Retrieved from \url{https://geographicdata.science/book/notebooks/09_gwr.html}.

\bibitem{fotheringham2002}
Fotheringham, A. S., Brunsdon, C., \& Charlton, M. (2002). \textit{Geographically Weighted Regression: The Analysis of Spatially Varying Relationships}. John Wiley \& Sons.

\bibitem{mgwr2019}
Oshan, T. M., Li, Z., Kang, W., Wolf, L. J., \& Fotheringham, A. S. (2019). mgwr: A Python implementation of multiscale geographically weighted regression for investigating process spatial heterogeneity and scale. \textit{ISPRS International Journal of Geo-Information}, 8(6), 269.

\bibitem{geopandas}
Jordahl, K., et al. (2020). geopandas/geopandas: v0.8.1. Zenodo. \url{https://doi.org/10.5281/zenodo.3946761}

\bibitem{statsmodels}
Seabold, S., \& Perktold, J. (2010). statsmodels: Econometric and statistical modeling with python. \textit{Proceedings of the 9th Python in Science Conference}.

\end{thebibliography}

\newpage
\begin{center}
    \section*{LAMPIRAN}
\end{center}
\addcontentsline{toc}{section}{LAMPIRAN}

Seluruh data penelitian, kode program (\textit{source code}), serta hasil keluaran pendukung lainnya dapat diakses melalui tautan Google Drive berikut:

\begin{center}
    \url{https://drive.google.com/drive/folders/1uobH4-HJMS4kfw8nuBLNdwAhVqs-ocYD?usp=sharing}
\end{center}

\end{document}
